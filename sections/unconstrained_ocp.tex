
\section{Unconstrained linear-quadratic OCP}

This section serves to explain the solution of the KKT system, which sits at the core of solving linear-quadratic optimal control problems.
Even optimal control problems in general.
Let's look at the unconstrained finite-horizon linear-quadratic optimal control problem, which is formulated as follows.

\[
\begin{aligned}
    & \underset{x_k, u_k}{\text{minimize}} & & \sum_{k=0}^{N} \frac{1}{2} \vxuone{k}\T \Hcoststacked{k} \vxuone{k} \\
    & \text{subject to} & & x_0 = \hat{x}\\
    & & & x_{k+1} = A_k x_k + B_k u_k + c_k\\
\end{aligned}
\]

Note that \textit{unconstrained} in this case means that there are no \textit{inequality} constraints on states or inputs.
The only constraints are the system dynamics and the initial state, but these are primal feasibility constraints, which means they must be satisfied in order to attain a feasible solution.
They are thus not a constraint on the problem in the sense of restricting the solution space.

\subsection{Solution via Karush-Kuhn-Tucker}

In terms of numeric optimization, the OCP above is an equlaity-constrained QP, which is convex and thus has a unique solution.
This solution is characterized by the Karush-Kuhn-Tucker (KKT) conditions.
To derive the KKT conditions, we first form the Lagrangian by introducing the Lagrange multipliers $\pi_k \in \R{n_x}$ for the dynamics constraints.

\[
\mathcal{L}(x_k, u_k, \pi_k) = \sum_{k=0}^{N-1} l_k(x_k, u_k) + l_N(x_N) + \sum_{k=0}^{N-1} \pi_{k+1}\T (A_k x_k + B_k u_k + c_k - x_{k+1})
\]

With the stage cost
\[
l_k(x_k, u_k) = \frac{1}{2} x_k\T Q_k x_k + \frac{1}{2} u_k\T R_k u_k + x_k\T S_k\T u_k + q_k\T x_k + r_k\T u_k
\]

and terminal cost
\[
l_N(x_N) = \frac{1}{2} x_N\T Q_N x_N + q_N\T x_N
\]

The KKT conditions are obtained by the stationary conditions, which says that the Lagrangian becomes stationary at the optimal solution with respect to all primal and dual variables.
The stationary conditions are given as follows.

Stationary w.r.t. states $x_k$ (for $k=0,\dots,N$):
\[
\nabla_{x_k} \mathcal{L} = Q_k x_k + S_k u_k + q_k + A_k\T \pi_{k+1} - \pi_k = 0
\]

Stationary w.r.t. inputs $u_k$ (for $k=0,\dots,N-1$):
\[ 
\nabla_{u_k} \mathcal{L} = R_k u_k + S_k\T x_k + r_k + B_k\T \pi_{k+1} = 0
\]

Stationary w.r.t. Lagrange multipliers $\pi_k$ (for $k=1,\dots,N$):
\[
\nabla_{\pi_k} \mathcal{L} = A_{k-1} x_{k-1} + B_{k-1} u_{k-1} + c_{k-1} - x_k = 0
\]

If we stack the decision variables and Lagrange multipliers into a single vector, we can write the KKT conditions as a large block-tridiagonal linear system.
This system obey a sparsity pattern, that can be exploited when solving it.
Here is an example showing the structure for $N=2$.

\[
\begin{bmatrix}
    Q_0   & S_0 &       &     &     & A_0\T &       \\
    S_0\T & R_0 &       &     &     & B_0\T &       \\
          &     & Q_1   & S_1 &     & -I    & A_1\T \\
          &     & S_1\T & R_1 &     &       & B_1\T \\
          &     &       &     & Q_2 &       & -I    \\
    A_0   & B_0 & -I    &     &     &       &       \\
          &     & A_1   & B_1 & -I  &       &       \\
\end{bmatrix}
\begin{bmatrix}
        x_0 \\ u_0 \\ x_1 \\ u_1 \\ x_2 \\ \pi_0 \\ \pi_1
\end{bmatrix}
= 
-\begin{bmatrix}
    q_0 \\ r_0 \\ q_1 \\ r_1 \\ q_N \\ c_0 \\ c_1
\end{bmatrix}
\]

The system can be summarized in the following form, which is the standard system arising from optimal control problems.\\

\[
\begin{bmatrix} \mathcal{H}  & \mathcal{A} \T \\ \mathcal{A}  & 0 \end{bmatrix} \begin{bmatrix} y \\ \pi \end{bmatrix} = \begin{bmatrix} -h \\ -c \end{bmatrix}
\]

where the Hessians $\mathcal{H}$ is block-diagonal and $\mathcal{A}$ is block-bidiagonal. \\

There are in general three approaches to solve this system.
The naive approach would be to feed this system as is to a dense cholesky solver.
This works fine, since the hessian is $\mathcal{H}\in\Sp{}$.
But this is is of order O(N3) ? and thus not relevant for practival applications.
The other two approaches are worth considering in a bit more detail.

\subsubsection{From KKT to Block-Cholesky}

The second approach is to notice the spacial sparsity pattern of the hessian $\mathcal{H}$ and realize, the the solution to the system obayes a block-diagonal structure.
To show this, let us attempt to solve this system forming the shur complement.
Eliminate y from the KKT system

\[
y = \mathcal{H}^{-1}(-h - \mathcal{C}\T \pi)
\]

and substitute into $\mathcal{C}y = -c$, we get

\[
\mathcal{C} \mathcal{H}^{-1} \mathcal{C}\T \pi = \mathcal{C} \mathcal{H}^{-1} h - c
\]

We now define $Y := \mathcal{C} \mathcal{H}^{-1} \mathcal{C}\T$ and $\beta := c - \mathcal{C} \mathcal{H}^{-1} h$, to arrive at the system in normal equations form.

\[
\begin{aligned}
    Y \pi &= \beta \\
    y &= \mathcal{H}^{-1}(-h - C\T \pi)
\end{aligned}
\]
Observing the structure of $Y$, we see that $\mathcal{H}$ is block-diagonal with the stage hessian blocks

\[
H_k = \begin{bmatrix} Q_k & S_k \\ S_k\T & R_k \end{bmatrix}
\]

$\mathcal{C}$ is block-bidiagonal, and thus $Y$ is block-tridiagonal

\[
Y = \begin{bmatrix}
    Y_{00}   & Y_{01}    &        &        \\
    Y_{01}\T & Y_{11}    & Y_{12} &        \\
            & Y_{12}\T  & Y_{22} & \ddots \\
            &           & \ddots & \ddots
\end{bmatrix}
\]

with $\Phi_k := H_k^{-1}$, the blocks of $Y$ are given as 

\[
Y_{k,k} = \begin{bmatrix} A_{k-1} & B_{k-1} \end{bmatrix} \Phi_{k-1} \begin{bmatrix} A_{k-1}\T \\ B_{k-1}\T \end{bmatrix} + \begin{bmatrix} I & 0 \end{bmatrix} \Phi_k \begin{bmatrix} I \\ 0 \end{bmatrix}
\]

\[
Y_{k,k+1} = -\begin{bmatrix} I & 0 \end{bmatrix} \Phi_k \begin{bmatrix} A_k\T \\ B_k\T \end{bmatrix}
\]

The next step is the cholesky factorization of $Y$
\[
Y = L_Y L_Y\T
\]

Due to the block-tridiagonal structure of $Y$, $L_Y$ is block-lower-bidiagonal:

\[
L_Y =
\begin{bmatrix}
    L_{00} &        &        & \\
    L_{10} & L_{11} &        & \\
           & L_{21} & L_{22} & \\
           &        & \ddots & \ddots
\end{bmatrix}
\]

This fact motivates a blockwise factorization with the following recursion

\[
L_{kk} L_{kk}\T = Y_{kk} - L_{k,k-1} L_{k,k-1}\T
\]

\[
L_{k+1,k} L_{kk}\T = Y_{k+1,k}
\]

Solve $L_Y z = \beta$ by forward subsditution, solve $L_Y\T \pi = z$ by backward substitution and finally, recover $y = H^{-1}(-h - C\T \pi)$.

\subsubsection{From KKT to Dynamic Programming}

The third approach is to notice, that the problem also has temporal structure.
To be fair, the spacial structure we exploited above stems exactly from that temporal structure, but the interpretation takes on a differnt route.
Let's take a closer look at the KKT conditions to see the temporal relationship.
The stationarity conditions with respect to $x_k$ can be rewritten as

\[
\pi_k = Q_k x_k + S_k u_k + q_k + A_k\T \pi_{k+1}
\]

This reveals that the dynamics of the costates $\pi_k$ propagate \textit{backward} in time. 
The stationarity conditions with respect to $\pi_k$ reveal that the dynamics of the states $x_k$ propagate \textit{forward} in time.

\[
x_{k+1} = A_k x_k + B_k u_k + c_k
\]

Both dynamics are coupled via the staionarity conditions for the inputs $u_k$

\[ 
R_k u_k + S_k\T x_k + r_k + B_k\T \pi_{k+1} = 0
\]



\textbf{Dynamic Programming and the Principle of Optimality}

Define the \textbf{optimal cost-to-go} (or value function) as the minimum cost 
achievable from state $x_k$ at time $k$:
\[
V_k(x_k) = \min_{u_k, \dots, u_{N-1}} \left\{ \sum_{j=k}^{N-1} l_j(x_j, u_j) + l_N(x_N) 
\;\middle|\; x_{j+1} = A_j x_j + B_j u_j + c_j \right\}
\]

Bellman's \textbf{principle of optimality} states that the value function satisfies 
the recursive relationship:
\[
V_k(x_k) = \min_{u_k} \left\{ l_k(x_k, u_k) + V_{k+1}(A_k x_k + B_k u_k + c_k) \right\}
\]
with terminal condition $V_N(x_N) = l_N(x_N)$.

The crucial insight for linear-quadratic problems is that if $V_{k+1}$ is quadratic 
in the state, then $V_k$ is also quadratic. Since the terminal cost $l_N$ is quadratic, 
by induction the value function is quadratic at all stages:
\[
V_k(x_k) = \frac{1}{2} x_k\T P_k x_k + p_k\T x_k + \rho_k
\]
where $P_k \in \R{n_x \times n_x}$ is symmetric positive semidefinite, 
$p_k \in \R{n_x}$, and $\rho_k \in \R{}$ is a scalar (which we can ignore for 
optimization purposes).

\subsection{Derivation of the Riccati Recursion}

Substituting the quadratic form into Bellman's equation and expanding:
\begin{align*}
V_k(x_k) = \min_{u_k} \Big\{ 
&\frac{1}{2} x_k\T Q_k x_k + \frac{1}{2} u_k\T R_k u_k + x_k\T S_k u_k + q_k\T x_k + r_k\T u_k \\
&+ \frac{1}{2}(A_k x_k + B_k u_k + c_k)\T P_{k+1} (A_k x_k + B_k u_k + c_k) \\
&+ p_{k+1}\T (A_k x_k + B_k u_k + c_k) + \rho_{k+1} \Big\}
\end{align*}

The expression inside the minimization is quadratic in $u_k$. 
Setting the gradient with respect to $u_k$ to zero:
\[
R_k u_k + S_k\T x_k + r_k + B_k\T P_{k+1}(A_k x_k + B_k u_k + c_k) + B_k\T p_{k+1} = 0
\]

Solving for $u_k$:
\[
u_k^* = -\underbrace{(R_k + B_k\T P_{k+1} B_k)^{-1}(S_k\T + B_k\T P_{k+1} A_k)}_{K_k} x_k 
- \underbrace{(R_k + B_k\T P_{k+1} B_k)^{-1}(r_k + B_k\T P_{k+1} c_k + B_k\T p_{k+1})}_{k_k}
\]

This gives the optimal control as an \textbf{affine state feedback}:
\[
\boxed{u_k^* = -K_k x_k - k_k}
\]

Substituting $u_k^*$ back into the Bellman equation and matching coefficients 
with $V_k(x_k) = \frac{1}{2} x_k\T P_k x_k + p_k\T x_k + \rho_k$ yields the 
\textbf{Riccati recursion}.


\newpage